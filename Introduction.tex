\section{Introduction}

This article illustrates preparation of ASME paper using \LaTeX2\raisebox{-.3ex}{$\epsilon$}. The \LaTeX\  macro \verb+asme2ej.cls+, the {\sc Bib}\TeX\ style file \verb+asmems4.bst+, and the template \verb+asme2ej.tex+ that create this article are available on the WWW  at the URL address \verb+http://iel.ucdavis.edu/code/+. To ensure compliance with the 2003 ASME MS4 style guidelines  \cite{asmemanual}, you should modify neither the \LaTeX\ macro \verb+asme2ej.cls+ nor the {\sc Bib}\TeX\ style file \verb+asmems4.bst+. By comparing the output generated by typesetting this file and the \LaTeX2\raisebox{-.3ex}{$\epsilon$} source file, you should find everything you need to help you through the preparation of ASME paper using \LaTeX2\raisebox{-.3ex}{$\epsilon$}. Details on using \LaTeX\ can be found in \cite{latex}. 

In order to get started in generating a two-column version of your paper, please format the document with 0.75in top margin, 1.5in bottom margin and 0.825in left and right margins.  Break the text into two sections one for the title heading, and another for the body of the paper.  

The format of the heading is not critical, on the other hand formatting of the body of the text is the primary goal of this exercise.  This will allow you to see that the figures are matched to the column width and font size of the paper.  The double column of the heading section is set to 1.85in for the first column, a 0.5in spacing, and 4.5in for the second column.  For the body of the paper, set it to 3.34in for both columns with 0.17in spacing, both are right justified. 

The information that is the focus of this exercise is found in 
section~\ref{sect_figure}.
Please use this template to format your paper in a way that is similar to the printed form of the Journal of Mechanical Design.  This will allow you to verify that the size and resolution of your figures match the page layout of the journal.  The ASME Journal of Mechanical Design will no longer publish papers that have the errors demonstrated here.

ASME simply requires that the font should be the appropriate size and not be blurred or pixilated, and that lines should be the appropriate weight and have minimal, preferably no, pixilation or rasterization.

The journal uses 10pt Times Roman Bold for headings, but Times Bold is good enough for this effort.  The text is set at 9pt Times Roman, and again Times will be fine.  Insert a new line after the heading, and two lines after each section.  This is not exactly right but it is close enough.
