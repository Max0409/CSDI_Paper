%%%%%%%%%%%%%%%%%%%%%%%%%%% asme2ej.tex %%%%%%%%%%%%%%%%%%%%%%%%%%%%%%%
% Template for producing ASME-format journal articles using LaTeX    %
% Written by   Harry H. Cheng, Professor and Director                %
%              Integration Engineering Laboratory                    %
%              Department of Mechanical and Aerospace Engineering    %
%              University of California                              %
%              Davis, CA 95616                                       %
%              Tel: (530) 752-5020 (office)                          %
%                   (530) 752-1028 (lab)                             %
%              Fax: (530) 752-4158                                   %
%              Email: hhcheng@ucdavis.edu                            %
%              WWW:   http://iel.ucdavis.edu/people/cheng.html       %
%              May 7, 1994                                           %
% Modified: February 16, 2001 by Harry H. Cheng                      %
% Modified: January  01, 2003 by Geoffrey R. Shiflett                %
% Modified: July 19, 2009 as a template in a single column for       %
%           ASME Journals by Harry H. Cheng                          %
% Use at your own risk, send complaints to /dev/null                 %
%%%%%%%%%%%%%%%%%%%%%%%%%%%%%%%%%%%%%%%%%%%%%%%%%%%%%%%%%%%%%%%%%%%%%%

%%% use 10pt options with the asme2ej format
\documentclass[10pt]{asme2ej}

\usepackage{epsfig} %% for loading postscript figures

%% The class has several options
%  onecolumn/twocolumn - format for one or two columns per page
%  10pt/11pt/12pt - use 10, 11, or 12 point font
%  oneside/twoside - format for oneside/twosided printing
%  final/draft - format for final/draft copy
%  cleanfoot - take out copyright info in footer leave page number
%  cleanhead - take out the conference banner on the title page
%  titlepage/notitlepage - put in titlepage or leave out titlepage
%  
%% The default is oneside, onecolumn, 10pt, final


\title{CSDI paper 
}

%%% first author
\author{Harry H. Cheng
    \affiliation{
	Professor, Fellow of ASME\\
	Integration Engineering Laboratory\\
	Department of Mechanical Engineering\\
	University of California\\
	Davis, California 95616\\
    Email: hhcheng@ucdavis.edu
    }	
}

%%% second author
%%% remove the following entry for single author papers
%%% add more entries for additional authors
\author{J. Michael McCarthy\thanks{Address all correspondence related to ASME style format and figures to this author.} \\
    \affiliation{ Editor, Fellow of ASME\\
	Journal of Mechanical Design\\
        Email: jmmccart@uci.edu
    }
}

%%% third author
%%% remove the following entry for single author papers
%%% add more entries for additional authors
\author{Third Co-author\\
        Graduate Research Assistan, Student Member of ASME\\
       {\tensfb Fourth Co-author}\thanks{Address all correspondence for other issues to this author.} 
    \affiliation{Title, Member of ASME\\
        Department or Division Name\\
        Company or College Name\\
        City, State (spelled out), Zip Code\\
        Country (only if not U.S.)\\
        Email address (if available)
    }
}


\begin{document}

\maketitle    

%%%%%%%%%%%%%%%%%%%%%%%%%%%%%%%%%%%%%%%%%%%%%%%%%%%%%%%%%%%%%%%%%%%%%%
\begin{abstract}
{\it 
This article illustrates preparation of the final version of
an ASME journal paper submitted for publication using 
\LaTeX2\raisebox{-.3ex}{$\epsilon$}. For the convenience of proofreading
and editing, the final version shall be formatted in a single column. 
This article is formatted based on the contents in the article
entitled ``{\rm An ASME Journal Article Created Using 
\LaTeX2\raisebox{-.3ex}{$\epsilon$} in ASME Format for 
Testing Your Figures,}" which is a template for 
preparation of ASME papers submitted for review.
An abstract for an ASME paper should be less than 150 words and is normally in italics.  Notice that this abstract is to be set in 9pt Times Italic, single spaced and right justified.  
%%% 
Please use this template to test how your figures will look on the printed journal page of the Journal of Mechanical Design.  The Journal will no longer publish papers that contain errors in figure resolution.  These usually consist of unreadable or fuzzy text, and pixilation or rasterization of lines.  This template identifies the specifications used by JMD some of which may not be easily duplicated; for example, ASME actually uses Helvetica Condensed Bold, but this is not generally available so for the purpose of this exercise Helvetica is adequate.  However, reproduction of the journal page is not the goal, instead this exercise is to verify the quality of your figures. 
}
\end{abstract}

%%%%%%%%%%%%%%%%%%%%%%%%%%%%%%%%%%%%%%%%%%%%%%%%%%%%%%%%%%%%%%%%%%%%%%
\begin{nomenclature}
\entry{A}{You may include nomenclature here.}
\entry{$\alpha$}{There are two arguments for each entry of the nomemclature environment, the symbol and the definition.}
\end{nomenclature}

The primary text heading is  boldface and flushed left with the left margin.  The spacing between the  text and the heading is two line spaces.

%%%%%%%%%%%%%%%%%%%%%%%%%%%%%%%%%%%%%%%%%%%%%%%%%%%%%%%%%%%%%%%%%%%%%%
\input{introduction}



\end{document}
