\subsection{DRAM Buffer Management}
If PC-DB commit a transaction, it will add the new records` version in the DRAM. What if the DRAM is full? There are several strategies to expel the records in the DRAM. One very straight way is to expel the records of a transaction as soon as the transaction is committed. However, if there are other transactions operating on the same records in the meanwhile, they has to wait for the background thread to fetch the versions of the records back to the DRAM, which will take rather a long time. Also, we can  add a reference counter to each of these records in the DRAM. If one transaction makes operations on the record, the reference counter will add one, at the same time, if one transaction is committed, the reference counter of the records it operates on will subtract one. When we need to expel some records, we can expel the records whose reference count is zero. However, this method will transform the read operation into a write operation, which costs a lot. To solve this problem, PC-DB has two approaches. One straight and effective way is to add a valid period for each of the record in the DRAM. When we need to expel some records, we choose these records who are time out. 

To manage data buffer, there are many algorithms. The most commonly used algorithm is the LRU algorithm that makes a simple assumption for all the data accesses: if a data block is accessed once, it will be accessed again. This locality metric is also called "recency'', which is implemented by a LRU stack. Each data access is recorded in the LRU stack, where the top entry is the most recent accessed (MRU) and the bottom entry is the least recent accessed (LRU). The LRU entry is the evicted item as the buffer is full (reflected by the full LRU stack). If data access pattern follows the simple assumption of LRU, the strong locality data set is well kept in the buffer by LRU. However, the LRU algorithm fails to handle the following three data access patterns: (1) each data block is only accessed once in a format of sequential scans. In this situation, the buffer would be massively polluted by weak or no locality data blocks. (2) For a cyclic (loop-like) data access pattern, where the loop length is slightly larger than the buffer size, LRU always mistakenly evicts the blocks that will be accessed soon in the next loop. (3) In multiple streams of data accesses where each stream has its own probability for data re-accesses, LRU could not distinguish the probabilities among the streams. 

Our second approach to evict records in the DRAM is based on the LIRS algorithm, which performs better than the LRU algorithm. The pages in the DRAM are divided into two kinds: the cold pages and the hot pages. The division is based on the recent operation frequency. When a new record needs to enter into the DRAM but there is no free space, PC-DB chooses a cold page and migrate the page that contains the new records to replace the cold page. In order to divide the pages in DRAM into cold pages and hot pages, LIRS maintains two sets: High Inter-reference Recency set and Low Inter-reference Recency set. Low Inter-reference Recency set contains the pages that are operated on (read and write) frequently within a period of time, While High Inter-reference Recency set contains the cold pages. In the LIRS algorithm, two parameters, IRR(inter-reference recency) and Recency, are used . IRR is the last two visit intervals of a page, and Recency is how many other pages have been visited since the last visit of the page. The IRR and Recency parameters do not contain the number of duplicate pages because the repeated calculation of the page does not have much effect on the priority of current page. The division of High Inter-reference Recency set and Low Inter-reference Recency set is based on the IRR, and if two pages have the same IRR, the page with the larger Recency is replaced. LIRS algorithm use stack S and list Q to manage the two set. Stack S is used to maintain the hot pages and the potential hot pages, and List Q is used to link all the cold pages. In this way, the three LRU issues, which are talked above, are addressed. 
