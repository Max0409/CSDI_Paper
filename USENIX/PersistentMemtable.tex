\section{Persistent Memtable}
In Rock-DB, MemTable is a skiplist stored in DRAM, which is not persistent. It adopts a Write Ahead Log(WAL) for system consistency. WAL includes the key -value pairs ,their version and their checksum. It is sequentially appended in the persistent storage before written in the MemTable . However, the overheads of WAL become the bottleneck of write operations and it takes rather long time to recover from crash. 

Based on the above problems, we adopt persistent memory for optimization. 
The memtable is stored in persistent memory ,which avoids the overheads of WAL for consistency and durability. 
We adopt  $mfence()$ to achieve instruction serialization and $clflush()$ to guarantee that data in cache is flushed into persistent memory and can be recovered under failure. 
Algorithm 1 shows the insert operation. Other operations such as update and delete can also be achieved by atomic write operation.

\begin{algorithm}[t]
    \caption{Insert(key, value, version,preNode)} %算法的名字
   
    
    currentNode:=new Node(key,value,version);
    
    mfence();
    
    currentNode.next=preNode.next;
    
    clflush(currentNode);
    
    mfence();
    
    preNode.next:=currentNode;
    
    clflush(preNode.next);
    
    mfence();

   
    \end{algorithm}

Once the system fails, 
we can simply recover the skiplist by locate the root of skiplist, 
the address of which is stored in the MANIFEST file. 
Because the data has already stored in persistent memory , 
it will be quick compared with recovering from WAL.