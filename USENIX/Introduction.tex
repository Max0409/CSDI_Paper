%-------------------------------------------------------------------------------
\section{Introduction}
%-------------------------------------------------------------------------------

Unlike the relational database, the key-value (KV) database does not need to know the data in the value, so it has higher flexibility. The value can be string, file or picture, and the storage content is diverse. Nowadays, KV stores is widely used in various data intensive applications, such as social network~\cite{RocksDB}, e-commerce~\cite{Dynamo} and network index~\cite{Bigtable}. Log structured merge trees (LSM-tree)~\cite{LSM-tree} is a common way to implement KV stores, such as BigTable~\cite{Bigtable}, LevelDB~\cite{LevelDB} and RocksDB~\cite{RocksDB}. LSM-tree is mainly aimed at write intensive and few query scenarios. The core idea is to give up part of the read performance in exchange for the maximum write performance. Efficient write performance is mainly achieved by constructing a buffer in memory to turn write requests into batch processing, thus turning random writes to disks into sequential writes. In order to guarantee the order of data and improve the speed of data access, LSM-tree implements a multi-level data structure, and maintains the data structure through the background thread merge-sort operation (compaction). But it also brings the problem of read and write amplification.

In order to ensure that the system can recover from the failure, Write-Ahead-Log (WAL) is needed to record the operation before writing to memory. After the data is persisted to disk, the log is deleted. Using WAL will cause write performance degradation, so by default, storage systems based on LSM-tree such as LevelDB and RocksDB disable WAL to achieve better performance, but this reduces data consistency, because when crash occurs, some data may be lost.

Because LSM-tree has the problem of read amplification~\cite{WiscKey,NVMKV,PebblesDB}, it is necessary to optimize the concurrency control protocol in KV stores based on LSM-tree. For example, in optimistic concurrency control, in order not to access the data of the disk, the verification phase will only be carried out in memory. If the version data is not in memory, the transaction will abort, which improves the abort rate of the transaction.

With the emergence of byte-addressable persistent memory (PM), there occurs new opportunities and research directions for improving the performance of KV stores. PM has recently become more and more involved in key-value stores \cite{NVMRocks,DBLP:conf/usenix/KannanBGAA18,DBLP:conf/usenix/XiaJXS17}, which ehances the performance of storage system. PM technologies such as PCM~\cite{PCM}, memristors and 3D XPoint~\cite{3DXPoint} promote the rapid development of PM. PM can not only replace disks such as HDDs and SSDs but also be connected via a memory bus and act like DRAM. PM is expected to achieve a comparable read performance with DRAM and the write latency of PM is about 10x higher than DRAM~\cite{DBLP:conf/usenix/XiaJXS17}, but the cost of PM is lower than DRAM. 

In this paper, we use PM to replace the memory components in the RocksDB architecture, and use persistent memory to store memtable and immutable memtable. Because PM has the characteristics of byte-addressability and persistence, it does not need Write-Ahead-Log, reducing the cost of log, and has a stronger consistency. In addition, it has faster recovery speed, because recovery does not need to scan log entry. However, modern CPUs have multiple caches, and in some cases, CPUs may reorder some memory write instructions to improve write performance, which may lead to inconsistent system state. Therefore, we use cacheline flush and memory fence instructions to guarantee the consistency of persistent memory.

In order to optimize the concurrency control of RocksDB, especially Optimistic Concurrency Control (OCC), we use DRAM-based cache to maintain the sequence number of outstanding transactions, so that we can validate the sequence number in DRAM-based cache when the version cannot be retrieved in memory during the validation phase, instead of accessing the disk. Since the sequence number is maintained by cache, it has no need to directly access disk and thus no need to abort transaction, which improves read amplification and reduces the ratio of transaction abort.

To summarize, we make the following contributions in this paper.
\begin{itemize}
\item We explain the trade-off between performance and consistency in RocksDB and use persistent memory to store memtable and immutable memtable, so it does not require Write-Ahead-Log, which reduces logging costs, has better consistency and faster recovery.
\item We show the limitation of Optimistic Concurrency Control in RocksDB, and use DRAM-based cache to improve read amplification and reduce transaction abort ratio.
\item Our expected evaluation results show that PC-DB provides higher read and write performance, faster recovery and lower transaction abort ratio compared with RocksDB.
\end{itemize}

The rest of this paper is organized as follows. Section 2 gives an introduction on persistent memory technologies for KV store and the structure and features of RocksDB with the limitation of WAL overhead and read amplification, explaining the motivation of our work. Section 3 discusses how to leverage persistent memory in RocksDB and presents the design details of DRAM-based cache. Section 4 shows the design of experiments and the expected experimental results. At last, Section 5 discusses the related work and Section 6 concludes the paper.