%-------------------------------------------------------------------------------
\section{Background and Motivation}
%-------------------------------------------------------------------------------
\subsection{Persistent Memory}
Persistent Memory, also known as Non-volatile memory(NVM), is a byte-addressable persistent storage device between DRAM and disk in the heterogeneous memory hierarchy. PM can be attached to a memory bus socket just like DRAM, which enables it to be accessed via load and store instructions. Modern PM technologies include phase change memory (PCM)~\cite{PCM}, memristors and 3D XPoint~\cite{3DXPoint}.

Embracing the feature of byte-addressability, PM achieves a comparable read performance with DRAM. However, the write latency of PM is about 10x higher than DRAM~\cite{DBLP:conf/usenix/XiaJXS17}, but the cost of PM is lower than DRAM. These properties make PM a suitable choice for replacing DRAM.

Since the PM can be connected via a memory bus with byte-addressible feature, the PM supports atomic writes of 8 bytes~\cite{DBLP:conf/fast/LeeLSNN17}. Compared with the traditional storage devices using block access, PM supports more fine-grained writes. When writing persistent data, we need to ensure that the data structure is consistent, even in the event of system crash. However, in some cases, modern CPUs may reorder some of the memory write instructions to improve write performance. To keep the PM write order and data structure consistent, we need to explicitly use instructions such as \textbf{MFENCE} and \textbf{CLFLUSH} (Intel x86)~\cite{SLMDB,DBLP:conf/fast/LeeLSNN17,DBLP:conf/usenix/KannanBGAA18} to make memory writes ordered and consistent. In addition, in the case where the size of the data written to the PM is greater than 8 bytes, if the system crashs and performs recovery, the recovered data structure may be partially updated, resulting in inconsistent state. Techniques such as logging and Copy-on-Write (CoW) can be used to handle this situation.


As a new storage device, PM offers new opportunities and research directions for optimizing KV stores. There are previous studies~\cite{NVMRocks,DBLP:conf/usenix/KannanBGAA18,DBLP:conf/usenix/XiaJXS17} that use PM in KV stores to optimize system performance, which requires to redesign the data structure, such as skiplist~\cite{DBLP:conf/usenix/KannanBGAA18, SLMDB}. In this work, in addition to redesigning RocksDB's skiplist, we also introduce DRAM-based cache, which works with PM collaborativly to optimize the performance of the system.
\subsection{RocksDB}